\documentclass[11pt,english,a4paper]{article}
\usepackage[right=3.6cm,left=3.6cm,bottom=3.6cm,top=3.6cm]{geometry}
\usepackage{mathpazo}
\renewcommand{\sfdefault}{lmss}
\renewcommand{\ttdefault}{lmtt}
\renewcommand{\familydefault}{\rmdefault}
\usepackage[T1]{fontenc}
\usepackage[utf8]{inputenc}
\setlength{\parskip}{\smallskipamount}
\setlength{\parindent}{0pt}
\usepackage{color}
\usepackage[english]{babel}
\usepackage{graphicx}
\usepackage{amsmath}
\usepackage{amsthm}
\usepackage{array}
\usepackage[unicode=true,bookmarks=true,bookmarksnumbered=true,
 bookmarksopen=true,bookmarksopenlevel=4,breaklinks=true,pdfborder={0 0 0},
 backref=false,colorlinks=false]{hyperref}
\hypersetup{pdftitle={SINF2345 Languages and Algorithms for Distributed Applications - Project report},pdfauthor={Martin Trigaux -- Bernard Paulus}}
\usepackage{breakurl}
\usepackage{todonotes}

%% DOCUMENT %%
\begin{document}
\title{SINF2345 Languages and Algorithms for Distributed Applications \\ Project report}
\author{
  \begin{tabular}{c c}
    Martin \textsc{Trigaux}  &  Bernard \textsc{Paulus} \\
    \small SINF22MS                 &  \small SINF22MS \\
    \small \href{mailto:martin.trigaux@student.uclouvain.be}{martin.trigaux@student.uclouvain.be}  &  
    \small \href{mailto:bernard.paulus@student.uclouvain.be}{bernard.paulus@student.uclouvain.be}
  \end{tabular}
}
\date{\emph{\today}}
\maketitle

\tableofcontents  
\section*{Introduction}
The aim of the project is to build a distributed banking system. % QUESTION : Usefull to remind the full mission ?
The system is distributed in a way such that any bank node is award of the full banking system state at all time but only atomics messages are transfered between nodes.

To build this project, we decided to use the programming language Erlang.
We believe this language is highly adequate for concurrent applications and that motivated us to use it for this project.
This language handles also very well events and made easier the implementation of proposed algorithm in the reference book.

\section{Architecture}
To realise the distributed system, we used the reference book ``Reliable and Secure Distributed Programming'' by C. Cachin, R. Guerraoui and L. Rodrigues second edition.
As the modules and algorithms are tested and well defined, we tried to stick as much as possible to the book and implemented several algorithms from it.
We used a architecture based on layers and encapsulation of messages.
Each module is connected to other and behave upon events (recieved messages from other modules).

\subsection*{Basic abstractions}
We used the \textbf{perfect link} module\footnote{Module 2.3 p37 in reference book} for basic point to point communications.
These are at the lowest level of our architecture and simply transmit messages to above modules.
The perfect link ensures \emph{reliability} and the \emph{no duplication} property.\\

To handle failing nodes in a partially synchronous system, we used an \textbf{eventual failure detector} module\footnote{Module 2.8 p54 in reference book}.
This was implemented using the \textbf{increasing timeout failure detector} algorithm\footnote{Algorithm 2.7 p55 in reference book}.\\

Once we have successfully detected faulty processes, we can focus on correct ones and elect a leader.
The \textbf{eventual leader detector} module\footnote{Module 2.9 p56 in reference book} allows to elect a reliable as a leader to perform certain computations on behalf of the others.
We implemented the \textbf{monarchical eventual leader detector} algorithm\footnote{Algorithm 2.8 p58 in reference book}.
This algorithm elect the leader with the highest rank among the alived nodes with the possibility to restore suspected nodes (eg: too slow to reply and were wrongly suspected as failing).


\subsection*{Broadcasts}
We used the \textbf{best effort broadcast} module\footnote{Module 3.1 p75 in reference book} for implement the broadcast messages between nodes.
This module transmit messages to the adequate perfect links.
As we want to handle failing nodes, we used a \textbf{reliable broadcast} algorithm\footnote{Module 3.2 p77 in reference book}.
To ensure the \emph{agreeement} property, we implemented the \textbf{eager reliable broadcast} algorithm\footnote{Algorithm 3.3 p80 in reference book}.


\subsection*{Consensus}
As we are working in a concurrent system, we need to reach to a \textbf{consensus} and decide the next state of the system.
We used the \textbf{epoch-change} abstraction\footnote{Module 5.3 p218 in reference book} to take a decision on a proposed value.
This was implemented using the \textbf{leader based epoch change} algorithm\footnote{Algorithm 5.5 p219 in reference book}.
However the book makes the assumption that the local stack of messages is fifo which may not be true at all time. \\ % explain more here

In the attempt to obtain a consensus, we used the \textbf{epoch consensus} module\footnote{Module 5.4 p221 in reference book}.
We implemented it using the \textbf{read write epoch consensus} algorithm\footnote{Algorithm 5.6 p223 in reference book}.
The algorithm uses timestamp in a state value so as to serves the \emph{validity} and \emph{lock-in} or the epoch consensus.
The method used relies on a majority if correct processes and assumes we have at least more than the half on non-failing nodes.
In the \textbf{leader driven consensus} algorithm\footnote{Algorithm 5.7 p225 in reference book}, we distinguish the instances of epoch consensus by their timestamp.
Once we receive a new epoch event and need to switch the epoch, the algorithm aborts the running epoch consensus and initialize the next epoch consensu using the state of the previous one.

\subsection*{Total order broadcast}
\textbf{consensus based total order broadcast}
\textbf{total order broadcast}

\section{Manual}

\section{Conclusion}
 
\end{document}
